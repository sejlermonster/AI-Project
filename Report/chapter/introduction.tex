\chapter{Introduction}
\label{chp:intro}

This report is a part of the project in the course \emph{AI in Robotics} at Aarhus University Department of Engineering.

In the report, the theory presented in the course is briefly explained. This includes different techniques for localization, path finding and control. These techniques are described in Part \ref{part:theory} of the report in chapters \ref{chp:local}, \ref{chp:kalman}, \ref{chp:partFilter}, \ref{chp:search}, \ref{chp:pid} and \ref{chp:slam}

In Part \ref{part:project} of the report the developed software for a robot is described. The implementation tries to solve \emph{Kidnapped robot problem}. The project is developed for a Lego Mindstorm robot.

The Lego Mindstorm robot includes two motors used for motion. A gyroscope is used calculating the angle of the robot. A Ultra sound sensor is used for calculating the distance to objects. The developed software for the robot implements a particle filter to localize the position of the robot. Furthermore A* is implemented for path finding.