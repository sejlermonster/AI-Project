\chapter{Particle Filter}
\label{chp:partFilter}

The Kalman filter described in chapter \ref{chp:kalman} used Gaussians to describe the belif of posterio. The particle filter instead represents this by a set of random samples. This enables it to represent the belief in a much broader space of distribution.

In a particle filter a set of particles are sampled and they all denote a possible true state at time $t$. Every particle include a likelihood of that particle being the true state. This means that the denser an area is of particles the more likely the true state is in that area.

The algorithm can be described with the following steps:
\begin{itemize}
	\item 1. Generate hypotheical state for time t 
	\item 2. Calculate importance weight based on measurement
	\item 3. Resample based on importance weight
\end{itemize}

First the particles are generated randomly. Then given a measurement the importance weight is calculated. This indicates how much a specific particle agrees with the measurement. The particles are then resampled. The resampling is done by creating a new set of replacement particles based on the importance weight. The higher the importance weight the higher probability of the particle being added multiple times.

They idea behind of the Particle filter is to represent the posterio by a set of random samples drawn.