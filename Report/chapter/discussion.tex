\chapter{Discussion}
\label{chp:disc}
In general the results show that the implementation was successful but different things limited the project and could have been improved. In this chapter these different aspects are discussed

The robots motion was done using only the gyroscope and the ultrasonic sensor. When turning it was based on the error between the gyroscope measurement and the the desired turning angle. When moving forward and backwards it was based on the erorr between the measured distance and the desired measured position. The desired measured position was defined from the desired movement distance. This way of controlling the robot gave some complications because the sensors were quite inaccurate. When the robot measured distance were the sensor was no perpendicular to the object the distance might jump between values such as 30 and 90. This made the robot oscillate in some situations. Similar for turning the gyroscope was very inaccurate and suddenly the measured angle might started counting upwards while the robot was standing still. Generally this lead to very imprecise motion in some situations while in other scenarios the robot might perform as expected. Some of these complications might have been been approached by implementing a Kalman Filter and thereby combining the measured value with the predicted motion. Choosing a good kalman gain would ensure that the measurements would only be given a certain weight in the calculation. This would theoretically have made the motion of the robot more precise.

Because only the proportional term of the PID controller it was chosen to move the robot very slowly and looking at the error very often. By doing so the overshoot were very small and rarely happen. But implementing a full PID controller would have allowed the robot to move more quickly around in the world

In general the project was limited by the Lego Mindstorm Ev3 system. The sensors were very inaccurate and using for example a LIDAR instead of a ultrasonic sensor would probably have yielded a better result. On the other hand an advantage of the Lego Mindstorm system was that it enabled the group to focus on creating the software presented in the course.

In general the Particle Filter and A* algorithm developed in the project perfomed quite well even though the motion and sensors were quite inaccurate. The implementation were simulated on a computer first and then tested on the robot. Using Bluetooth it was possible to follow the robot and see how the particles were places on the map and what path A* suggested. By doing so it was easier to debug and see if the algorithm performed as expected.