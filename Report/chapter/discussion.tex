\chapter{Discussion}
\label{chp:disc}
The results shows how the implementation was successful. It is important to note though that different things limited the project and could have been improved. In this chapter these different aspects are discussed.

The robot's motion was done using only the gyroscope and the ultrasonic sensor. When turning it was based on the error between the gyroscope measurement and the the desired turning angle. When moving forward and backwards it was based on the error between the measured distance and the desired measured position. The desired measured position was defined from the desired movement distance. This way of controlling the robot gave some complications because the sensors were quite inaccurate. When the robot measured distance where the sensor was not perpendicular to the object the distance might jump between values such as 30 and 90. This made the robot oscillate in some situations. Similar for turning the gyroscope was very inaccurate and suddenly the gyroscope could start drifting. Generally this lead to very imprecise motion in some situations while in other scenarios the robot might perform as expected. Some of these complications might have been been approached by data fusion between multiple sensors. This could have created a more certain measurement and thereby avoided very imprecise measurements. A full PID controller could also have been implemented. This would also theoretically have helped against a systematic bias between the two motors that was also detected. Because only the proportional term of the PID controller was implemented, it was chosen to move the robot very slowly and looking at the error very often. By doing so the overshoot were very small and rarely happened. But implementing a full PID controller would have allowed the robot to move more quickly around in the world.

Furthermore the project was limited by the Lego Mindstorm Ev3 system. The sensors were very inaccurate and using for example a LIDAR instead of an ultrasonic sensor would probably have yielded a better result. On the other hand an advantage of the Lego Mindstorm system was that it enabled the group to focus on creating the software presented in the course.

In general the Particle Filter and A* algorithm developed in the project performed well even though the motion and sensors were quite inaccurate. The implementation were simulated on a computer first and then tested on the robot. Using Bluetooth it was possible to follow the robot and see how the particles were placed on the map and what path A* suggested. By doing so it was easier to debug and see if the algorithm performed as expected.