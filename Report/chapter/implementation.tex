\chapter{Implementation}
\label{chp:impl}

\section{Structure and design}


\section{Particle Filter}
In this project a particle filter was implemented and used for localiation of the robot.

The particle filter is initialized with 3 parameters: the number of particles to be randomly generated and the length on the x and y axis.

\begin{lstlisting}[caption={Initialization of the particle filter},             _particleFilter.GenerateParticles(_numberOfParticles, world.GetLength(0), world.GetLength(1));
\end{lstlisting}

After the particles are generated the Particle filter can be used for moving around the particles and resampling. The resampling is implemented as explained in chapter \ref{chp:partFilter} with the resampling wheel algorithm. The implemented resampling function is shown in \emph{BLABLA}. The method is called with a measurement. First a method is called for calculating the importance weight of each particle. This function looks at each particle and calculates the distance to an object in the orientation of the particle. This distance is then 

\begin{lstlisting}[caption={Resampling wheel},               public void Resampling(double measurement)
{

SetImportanceWeight(measurement);
var particles2 = new List<Particle>();
var index = random.Next(0, Particles.Count);
var beta = 0.0;
var maxParticleWeight = Particles.Max(x => x.Weight);
var max = maxParticleWeight;
for (int i = 0; i < Particles.Count; i++)
{
beta = beta + random.NextDouble() * 2*max;
while (beta > Particles[index].Weight)
{
beta -= Particles[index].Weight;
index = (index + 1) % Particles.Count;
}
particles2.Add(Particles[index].Clone());
}
Particles = particles2;
NormalizeParticleWeights();
}
\end{lstlisting}


\subsection{Simulation}
This section present a simple simulation of the particle filter. In the example the simulated robot is placed in the world at coordinates (23,64) and will first scan the area, then drive forward twice and scan the area again. When the simulated robot drives forward it also measures the distance in that direction. This is similar to how the lego mindstorm is implemented.

In figure \ref{fig:sim1} the world is initialized. The red dot is the robot, the direction of the robot is displayed by the small line drawn from the square. The particles are the black dots. The landmarks placed in the world are colored green, orange and blue. In the illustration the particles are spread out uniformly in the world. In figure \ref{fig:sim2} the robot has made its first 360 degree scan and taken measurement every 90 degree. The particles are now centered around a few specific places in the world. The location is still very uncertain, this is mainly caused by the symmetry of the world.

\mySubFigure{Simulation/1_Initialized.PNG}{Simulation/2_afterScan.PNG}{}{Intialized particles and world}{Robot makes a 360 degree scan}{fig:sim}{fig:sim1}{fig:sim2}

In figure \ref{sim1} and \ref{sim2} the robot drives forward 10 centimeters twice. The robot takes a measurement after each 10 centimeters. As illustrated the particles are now beginning to be concentrated around the simulated robot.

\mySubFigure{Simulation/3_afterMoveAndMeas.PNG}{Simulation/4_afterMoveAndMeas.PNG}{}{Robot moves forward and measures distance}{Robot moves forward again and measures distance}{fig:sim}{fig:sim1}{fig:sim2}

In figure \ref{fig:5} the robots makes 360 degree scan and takes measurement every 90 degree. After the scan the location is very certain and the particles are all located around the simulated robot.

\myFigure{Simulation/5_afterScan.PNG}{Robot makes a 360 degree scan}{fig:5}{0.4} 

\section{A star}


\section{Combining Particle Filter and A star}